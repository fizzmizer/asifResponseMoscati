\documentclass[a4paper,11pt]{article}
\usepackage[utf8]{inputenc}
\usepackage{mathtools}
\usepackage{amsmath}
\usepackage{amsfonts}
\usepackage{amssymb}
\usepackage[english]{babel}
\usepackage{geometry}
%\usepackage{cite}
\usepackage{mathrsfs}
\usepackage{tikz-cd}
\usetikzlibrary{arrows}
\usepackage{natbib}
\usepackage{hyperref}
\usepackage{fancybox}
\usepackage{array}
\usepackage{lscape}
\usepackage{longtable}
\usepackage{multirow}
\usepackage[strict]{changepage}
\usepackage{authblk}

\usepackage{caption}
\captionsetup[figure]{font=scriptsize}


\title{TBA}
\date{}
\author[1]{Antoine Brandelet}
\author[1,2]{Jeremy Attard}

\affil[1]{Department of Philosophy and History of Science, University of Mons, Belgium}
\affil[2]{Department of Sciences, Philosophies and Societies, University of Namur, Belgium}

\begin{document}
\maketitle

\begin{abstract}
    In a recent paper, Moscati provides an analysis of as-if modelling in economics, more precisely in decision theory. According to him, most of decision theoretical models---be it neoclassical, behavioural or heuristic---are as-if, and that should be no worry at all. That observation is then taken to be a reason to move forward a realist understanding of modelling and embrace a form of instrumentalism.

    Here, we wish to support Moscati's claim of the positive epistemic value of as-if modelling while keeping a realist interpretation. We show that the core of the instrumentalist critique he provides is based on a misunderstanding of the very mechanism of as-if modelling. We then propose a realist framework in which the variety of modelling strategies, in economics as well as in physics, can be described.
\end{abstract}


\section{Introduction}
In a recent paper \citep{Moscati2023}

\bibliography{biblio} 
\bibliographystyle{plainnat}
\end{document}
