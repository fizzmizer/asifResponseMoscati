\documentclass[a4paper,11pt]{article}
\usepackage[utf8]{inputenc}
\usepackage{mathtools}
\usepackage{amsmath}
\usepackage{amsfonts}
\usepackage{amssymb}
\usepackage[english]{babel}
\usepackage{geometry}
%\usepackage{cite}
\usepackage{mathrsfs}
\usepackage{tikz-cd}
\usetikzlibrary{arrows}
\usepackage{natbib}
\usepackage{hyperref}
\usepackage{fancybox}
\usepackage{array}
\usepackage{lscape}
\usepackage{longtable}
\usepackage{multirow}
\usepackage[strict]{changepage}
\usepackage{authblk}

\usepackage{caption}
\captionsetup[figure]{font=scriptsize}


\title{TBA}
\date{}
\author[1]{Antoine Brandelet}
\author[1,2]{Jeremy Attard}

\affil[1]{Department of Philosophy and History of Science, University of Mons, Belgium}
\affil[2]{Department of Sciences, Philosophies and Societies, University of Namur, Belgium}

\begin{document}
\maketitle

\begin{abstract}
    In a recent paper, Moscati provides an analysis of as-if modelling in economics, more precisely in decision theory. According to him, most of decision theoretical models---be it neoclassical, behavioural or heuristic---are as-if, and that should be no worry at all. That observation is then taken to be a reason to move forward a realist understanding of modelling and embrace a form of instrumentalism.

    Here, we wish to support Moscati's claim of the positive epistemic value of as-if modelling while keeping a realist interpretation. We show that the core of the instrumentalist critique he provides is based on a misunderstanding of the very mechanism of as-if modelling. We then propose a realist framework in which the variety of modelling strategies, in economics as well as in physics, can be described.
\end{abstract}


\section{Introduction}
In a recent paper \citep{Moscati2023}... 

Moscati first distinguishes three classes of models in decision theory: neo-classical, behavioral and heuristic. The major representative of the first class is rational choice theory, or expected utility theory, in which agents are assumed to attach a utility value (a real number) to each outcome of possible risky choices, and to make the choice which maximizes the expected utility. The second class contains attempts to improve neo-classical models by various means, including more complex assumptions about the form of utility function or accounting for biased perceived probabilities of different outcomes by the decision makers. The third class also aims at amending neo-classical models but unlike the two first ones does not assume that agents maximize anything, but only that they make choices among a finite set of possibilities following some heuristic (simple) rules. Behavioral and heuristic models historically developed in reaction to epistemological criticisms addressed to neo-classical models. The most salient is the lack of realisticness of their basic assumptions due to the accumulating amount of evidence that these basic assumptions seem to be systematically violated in decision theory experiments. That is to say, the first class of models was originally criticized for it seems to commit to an instrumentalist philosophical position known as the ``as-if'' account of models in economics, mainstream in this field from the seminal work of Milton Friedman \cite{Friedman1953}. Moscati gives the following definition of the as-if account of modeling in decision theory \citep[p. 2]{Moscati2023}: 

\begin{quote}
Broadly speaking, as-if models attempt to account for the observable choices that
individuals make, but do not pretend to capture the underlying psychological
mechanisms that might generate those choices. Some underlying choice -
generating mechanism, such as utility maximization, is attached to the model.
However, in the as-if approach the decision theorist is agnostic about whether
this mechanism actually operates in the mind of the decision maker. She may
even deem, and explicitly acknowledge, that the posited mechanism and its
components (such as the utility function, the preference relation or the heuristic
rules), are only fictional constructs. Nonetheless, the decision theorist explains,
describes or predicts the decision maker’s choices as if they were generated by
the posited psychological mechanism at issue. Insofar as the as-if model is
capable of accounting for the decision maker’s choice behaviour or indicating ways to effectively control it for economic policy purposes, the model is considered
scientifically valid.
\end{quote}


The two other classes of models are made to be more acceptable from an epistemological viewpoint for they rest on human behavior's assumptions which correspond more to what is otherwise observed in experiments.

In his paper, Moscati defends two main positions: 1/ all decision theory models, including behavioral and heuristic ones, are (maybe more complex) as-if models -- whatever they claim to be; 2/ there is no worry at all with this situation, as long as it is seen within the right antirealist framework. According to him, the latter goes beyond a mere instrumentalism in which it does see explanation as a genuine and important aim of science. Moscati only argues, resting e.g. on Alisa Bokulich's work about models explanation \citep{Bokulich2009}, that it is possible to defend a (mechanism-based) view of explanation without being committed to any form of realism going beyond ontological realism. 


\bibliography{biblio} 
\bibliographystyle{plainnat}
\end{document}
